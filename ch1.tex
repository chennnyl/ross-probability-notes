\section{Combinatorial Analysis}
\subsection{Counting}

\begin{bdef}{The basic principle of counting}
    Suppose that two experiments are to be performed. Then, if experiment 1 can result in any one of $m$ possibilities, and if, for each outcome of experiment 1, experiment 2 can result in any one of $n$ outcomes, then together there are $mn$ possible outcomes of the two experiments.
\end{bdef}

\begin{changebar}
    \begin{example}
        A small community consists of 10 women, each of whom has 3 children. If one woman and one of her children are to be chosen as mother and child of the year, how many different choices are possible?
    \end{example}
    \begin{solution}
        By regarding the choice of woman as the outcome of experiment 1 and the choice of one of her children as the outcome of experiment 2, the basic principle of counting says there are $10\cdot3=30$ possible choices.
    \end{solution}
\end{changebar}

\begin{bdef}{The generalized basic principle of counting}
    If the experiments $e_1, e_2, \dots, e_r$ are to be performed such that $e_1$ may result in any of $n_1$ possible outcomes, $e_2$ may result in any of $n_2$ possible outcomes, etc., then there is a total of $n_1\cdot n_2 \cdot \cdots \cdot n_r$ outcomes for the $r$ experiments. 
\end{bdef}

\begin{changebar}
    \begin{example}
        A college planning committee consists of 3 freshmen, 4 sophomores, 5 juniors, and 2 seniors. A subcommittee of 4 people, consisting of 1 person from each class, is to be chosen. How many different subcommittees are possible?
    \end{example}
    \begin{solution}
        By regarding the choice of a subcommittee as the combined outcome of the four separate experiments of choosing a single representative from each class, then the generalized basic principle of counting says there are $3\cdot 4 \cdot 5 \cdot 2 = 120$ possible subcommittees.
    \end{solution}
\end{changebar}
    
\begin{changebar}
    \begin{example}
        How many different 7-character license plates are possible if the first 3 places are to be occupied by letters and the final 4 by numbers?
    \end{example}
    \begin{solution}
        $26^3 \cdot 10^4 = 175{,}760{,}000$ possible license plates.
    \end{solution}
\end{changebar}

\begin{changebar}
    \begin{example}\label{funccex}
        How many functions defined on $n$ points are possible if the range of the functions is $\left\{ 0, 1 \right\}$? 
    \end{example}
    \begin{solution}
        As each function can assign one of $2$ values to each of the $n$ points, the generalized basic principle of counting says there are $2^n$ possible functions.
    \end{solution}
\end{changebar}

\pagebreak

\subsection{Permutations}
\begin{bdef}{Permutations of unique objects}
    For $n$ unique objects, a particular ordered arrangement of the objects is known as a \textbf{permutation}. There are $n!$ unique such permutations.
\end{bdef}

\begin{changebar}
    \begin{example}
        How many different batting orders are possible for a baseball team of 9 players?
    \end{example}
    \begin{solution}
        $9! = 362{,}880$ possible orders.
    \end{solution}
\end{changebar}

\begin{changebar}
    \begin{example}
        A class in probability theory consists of 6 men and 4 women. After an exam, the students are ranked according to their performance. Assume that no two students obtain the same score. \begin{enumerate}[label=(\alph*)]
            \item How many different rankings are possible?
            \item If the men and women are ranked separately, how many different rankings are possible?
        \end{enumerate}
    \end{example}
    \begin{solution}\hfill
        \begin{enumerate}[label=(\alph*)]
            \item Each ranking corresponds to a permutation of the students, so there are $10! = 3{,}628{,}800$ possible rankings.
            \item There are $6! = 720$ possible rankings of the men and $4! = 24$ possible rankings of the women. From the basic principle of counting, there are then $720\cdot 24 = 17{,}280$ overall possible rankings.
        \end{enumerate}
    \end{solution}
\end{changebar}

\begin{changebar}
    \begin{example}
        Ms. Jones has 10 books she wants to arrange on her bookshelf. Of these, 4 are math books, 3 are chemistry books, 2 are history books, and 1 is a language book. Ms. Jones wants to arrange her books so that all the books dealing with the same subject are together on the shelf. How many different arrangements are possible.
    \end{example}
    \begin{solution}
        We can consider counting the possible permutations within each subject, then permute the subjects as groups on the shelf. There are $4!$ possible permutations of the math books, $3!$ permutations of the chemistry books, $2!$ permutations of the history books, and $1!$ permutations of the language book. Then there are $4!$ orderings of the subjects, each of which has $4! \cdot 3! \cdot 2! \cdot 1!$ orderings of the books within, giving us a total of $4!\cdot(4!\cdot3!\cdot2!\cdot1!) = 6{,}912$ orderings.
    \end{solution}
\end{changebar}

\begin{bdef}{Permutations with repetition}\label{permutationsrepetition}
    For $n$ objects, of which $n_1$ are alike, $n_2$ are alike, etc., through $n_r$ alike objects, there are \[
        \frac{n!}{n_1!n_2!\cdots n_r!}    
    \] unique permutations.
\end{bdef}

\begin{changebar}
    \begin{example}
        A chess tournament has 10 competitors, of which 4 are Russian, 3 are from the United States, 2 are British, and 1 is Brazilian. If the tournament result lists only the nationalities of the players in the order in which they placed, how many outcomes are possible?    
    \end{example}
    \begin{solution}
        There are \[
            \frac{10!}{4!3!2!1!} = 12{,}600
            \] possible outcomes.
        \end{solution}
    \end{changebar}

\subsection{Combinations}
\begin{bdef}{$n$ choose $k$}
    We define ${n \choose k}$ (read ``$n$ choose $k$''), for $k \leq n$, as \[
        {n \choose k} = \frac{n(n-1)(n-2)\cdots(n-k)}{k!} = \frac{n!}{k!(n-k)!}.
    \] ${n \choose k}$ represents the number of possible unordered combinations of $n$ objects taken $k$ at a time. By convention, ${n \choose k} = 0$ when $k > n$ or $k < 0$.
\end{bdef}

\begin{changebar}
    \begin{example}
        A committee of 3 people is to be formed from a group of 20 people. How many different committees are possible?
    \end{example}
    \begin{solution}
        There are $\displaystyle {20 \choose 3} = \frac{20!}{3!(20-3)!} = \frac{20\cdot 19 \cdot 18}{3 \cdot 2 \cdot 1} = 1{,}140$ possible committees.
    \end{solution}
\end{changebar}

\begin{changebar}
    \begin{example}
        From a group of 5 women and 7 men, \begin{enumerate}[label=(\alph*)]
            \item How many committees consisting of 2 women and 3 men can be formed?
            \item What if two of the men are feuding and refuse to serve on the committee together?
        \end{enumerate}
    \end{example}
    \begin{solution}\hfill
        \begin{enumerate}[label=(\alph*)]
            \item From the basic principle, there are $\displaystyle {5 \choose 2}{7 \choose 3} = 350$ possible committees of 2 women and 3 men.
            \item There are $\displaystyle {7 \choose 3} = 35$ total possible groups of men. To count just the groups which contain the two feuding men, we can simply count the number of ways to group $3-2 = 1$ of the $7-2 = 5$ remaining men: $\displaystyle {5 \choose 1} = 5$. Thus, removing the groups with the feuding pair, there are $35 - 5 = 30$ valid groups of men. The number of valid groups of women has not changed, so there are $\displaystyle 30\cdot{5 \choose 2} = 300$ possible committees.
        \end{enumerate}
    \end{solution}
\end{changebar}

\begin{changebar}
    \begin{example}
        Consider a set of $n$ antennae, of which $m$ are defective and $n-m$ are functional. Assume that defective antennae are indistinguisible among themselves, as are functional antennae. How many orderings of the antennae are there that do not contain two consecutive defectives?
    \end{example}
    \begin{solution}
        Imagine lining up the $n-m$ functional antennae. If no two defective antennae are to be consecutive, then the spaces between functional antennae must contain at most one defective antenna. Then there are $n-m+1$ possible positions between the functioning antennae, of which we must select $m$ to contain defective antennae. For example, for $n = 5$ and $m = 2$, a defective antennae could occupy any of the spots represented by a \texttt{|} in the following diagram:
        \begin{center}
            \texttt{|F|F|F|}
        \end{center}
        Then there are \[
            {{n-m+1} \choose m}    
            \] possible orderings in which there is at least one functional antenna between any two defective ones. 
        \end{solution}
\end{changebar}
\begin{bdef}{Pascal's identity}\label{pid}
    \[
        {n \choose k} = {{n-1} \choose {k - 1}} + {{n - 1} \choose k}    
    \] for $1 \leq k \leq n$.
\end{bdef}
\begin{proof}
    Consider a group of $n$ objects. For a given object (say $n_i$), there are $\displaystyle {n-1 \choose k-1}$ groups of size $k$ that contain $n_i$. There are also $\displaystyle {n-1 \choose k}$ groups of size $k$ that do \emph{not} contain $n_i$. Since these two sets of groupings are disjoint and comprise every possible grouping of size $k$ for the $n$ objects, it must be the case that $\displaystyle {n \choose k} = {n-1 \choose k-1} + {n-1 \choose k}$.
\end{proof}

\begin{bdef}{Selection from categories}
    There are \[
        {{n+k-1} \choose k}    
    \] ways to choose $k$ objects from $n$ different categories.
\end{bdef}

\begin{changebar}
    \begin{example}
        Imagine you're shopping for ice cream. There are 5 flavors to choose from, and you want to buy 15 pints. How many different selections of ice cream can you make?
    \end{example}
    \begin{solution}
        There are \[
            {{15 + 5 - 1} \choose 15} = {19 \choose 15} = 3{,}876    
            \] possible selections.
        \end{solution}
    \end{changebar}

\pagebreak
\subsection{Binomial coefficients}
The values ${n \choose k}$ are often referred to as \emph{binomial coefficients} because of their prominence in the binomial theorem.
\begin{bdef}{Binomial coefficients}
    In the binomial expansion of $(a+b)^n$, the coefficient of the $k$th term (starting at $0$) is equal to $\displaystyle {n \choose k}$. Alternatively, you can consider ${n \choose k}$ to be the coefficient of the term in $(a+b)^n$ in which $b$ is raised to the power $k$.
\end{bdef}

\begin{changebar}
    \begin{example}
        Consider \[
            (a + b)^3 = a^3 + 3a^2b + 3ab^2 + b^3.    
            \] The coefficient of the third term ($k = 2$) is $3$. ${3 \choose 2} = 3$.
        \end{example}
    \end{changebar}

\begin{bdef}{The binomial theorem}\label{bth}
    \[
        \begin{aligned}
            (x + y)^n &= \sum^n_{k = 0} {n \choose k} x^{n-k}y^k \\
            &= \sum^n_{k=0} {n \choose k} x^ky^{n-k}
        \end{aligned}
    \]
\end{bdef}
Here are two proofs of this theorem:
\begin{proof}
    Proof by induction:
    \begin{itemize}
        \item \textbf{Base case:}
            Let $n = 1$. Then $(x+y)^n = x + y$ and \[
                \sum^n_{k=0}{n \choose k} x^{n-k}y^{k} = {1 \choose 0} x^{1-0}y^0 + {1 \choose 1}x^{1-1}y^1 = x + y.    
            \]
        \item \textbf{Inductive step:} Assume that \nameref{bth} holds for some $n-1$, where $n \geq 1$. Then \[
            \begin{aligned}
                (x+y)^{n} &= (x+y)(x+y)^{n-1} \\ 
                &= (x+y)\sum^{n-1}_{k=0}{n-1 \choose k}x^{n-1-k}y^k \\
                &= x\left(\sum^{n-1}_{k=0}{n-1 \choose k} x^{n-k-1}y^k\right) + y\left(\sum^{n-1}_{k=0}{n-1 \choose k} x^{n-1-k}y^k\right) \\
                &= \sum^{n-1}_{k=0} {n-1 \choose k} x^{n-k}y^k + \sum^{n-1}_{k = 0}{n-1 \choose k}x^{n-k-1}y^{k+1}
            \end{aligned}
        \] Shift the bounds of the right sum forward by $1$ (substitute $i = k+1$): \[
            \begin{aligned}
                (x+y)^{n} &= \sum^{n-1}_{i=0} {n-1 \choose i}x^{n-i}y^i + \sum^{n}_{i = 1} {n-1 \choose i - 1} x^{n-i}y^i \\ 
                &= \left( {n-1 \choose 0} x^ny^0 \right) + \sum^{n-1}_{i=1} {n-1 \choose i} x^{n-i}y^i + \left( {n - 1 \choose n - 1} x^{0}y^n \right) + \sum^{n-1}_{i=1} x^{n-i}y^i \\
                &= x^n + y^n + \sum^{n-1}_{i=1} \left[ {n-1 \choose i-1} + {n - 1 \choose i} \right] x^{n-i}y^i,
            \end{aligned}
        \] which by \nameref{pid} is equal to \[
            x^n + y^n + \sum^{n-1}_{i=i} {n \choose i} x^{n-i}y^i.
        \] By incorporating the loose left terms and adjusting the bounds, we obtain \[
            \sum^{n}_{i=0} {n \choose i} x^{n-i}y^i,    
        \] as desired.
    \end{itemize}
\end{proof}
\begin{proof}
    Proof by a combinatorial argument:
    
    \vspace{5pt}
    
    \noindent Consider the product \[
        (x_1 + y_1)(x_2 + y_2)\cdots(x_n + y_n).    
    \] This expands to the sum of $2^n$ terms, each being the product of $n$ factors. Further, each of those $2^n$ terms will contain as a factor either $x_i$ or $y_i$ for each $i = 1, 2, \dots, n$. For example: \[
        (x_1 + y_1)(x_2 + y_2)(x_3 + y_3) = x_1x_2x_3 + x_1x_2y_3 + x_1y_2x_3 +x_1y_2y_3 + y_1x_2x_3 + y_1x_2y_3 + y_1y_2x_3 + y_1y_2y_3.    
    \] Now consider how many of the $2^n$ terms of this sum will have a given $k$ of the $x_i$s and $(n - k)$ of the $y_i$s as factors. This corresponds to a combination of $k$ members of the $n$ values $x_1, x_2, \dots, x_n$, so there are ${n \choose k}$ such terms. Then if we set all of the $x_i$s equal to $x$ and all of the $y_i$s equal to $y$, we can see that \[
        (x+y)^n = \sum^n_{k=0}{n \choose k}x^ky^{n-k},   
    \] which is the other of the two equivalent definitions of the binomial theorem from \autoref{bth}.
\end{proof}

\begin{changebar}
    \begin{example}
        How many subsets are there of a set $S$ containing $n$ elements?
    \end{example}
    \begin{solution}
        Since there are ${n \choose k}$ subsets of size $k$, we can use the binomial theorem in a surprising way: \[
            \sum^n_{k=0}{n \choose k} = (1 + 1)^n = 2^n.
            \]
            Alternatively, consider \autoref{funccex}. We can consider our $n$ points to be each element of $S$. Then, we can form a bijection between subsets of $S$ and each function as described in the problem: each function $f$ corresponds to the subset $S'$ such that $f(s) = 1 \iff s \in S'$. Proof that this is a bijection is left to the reader, but as we showed in \autoref{funccex} that there are $2^n$ such functions, there are therefore $2^n$ subsets of $S$.
        \end{solution}
    \end{changebar}

\pagebreak
\subsection{Multinomial coefficients}
\begin{bdef}{Multinomial coefficients}
    For $n_1 + n_2 + \dots + n_r = n$, we define $\displaystyle {n \choose n_1,n_2,\dots,n_r}$ to be \[
        {n \choose n_1,n_2,\dots,n_r} = \frac{n!}{n_1!n_2!\cdots n_r!}.
    \] ${n \choose n_1, n_2, \dots, n_r}$ represents the number of possible divisions of $n$ distinct objects into $r$ distinct groups of respective sizes $n_1, n_2, \dots, n_r$.
\end{bdef}

\begin{changebar}
    
    \begin{example}
        A police department in a small city consists of 10 officers. If the department policy is to have 5 of the officers patrolling the streets, 2 of the officers working full time at the station, and 3 of the officers on reserve at the station, how many different divisions of the 10 officers into the 3 groups are possible?
    \end{example}
    \begin{solution}
        There are ${10 \choose 5,2,3} = \frac{10!}{5!2!3!} = 2{,}520$ possible divisions.
    \end{solution}
\end{changebar}
    
\begin{changebar}
    \begin{example}
        In order to play a game of basketball, 10 children at a playground divide themselves into two teams of 5 each. How many different divisions are possible?
    \end{example}
    \begin{solution}
        There are ${10 \choose 5,5} = \frac{10!}{5!5!} = 252$ possible groupings, but as the ``order'' of the two teams is irrelevant (that is, the first and second groups of 5 are non-distinct), there are actually \[
            \frac{{10 \choose 5,5}}{2!} = \frac{252}{2} = 126    
            \] different divisions.
        \end{solution}
\end{changebar}

\begin{bdef}{The multinomial theorem}
    \[\begin{aligned}
        &(x_1+x_2+\dots+x_r)^n = \\
        &\sum_{(n_1,\dots,n_r):\: n_1+\dots+n_r = n} {n \choose n_1,n_2,\dots,n_r} x_1^{n_1}x_2^{n_2}\dots x_r^{n_r}
    \end{aligned}\]
    That is, the sum is over all sets of nonnegative integer vectors $(n_1, \dots, n_r)$ such that $n_1 + n_2 + \dots + n_r = n$.
\end{bdef}